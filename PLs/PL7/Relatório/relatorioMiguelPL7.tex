\documentclass[a4paper,10pt,twocolumn]{article}

\usepackage{booktabs}
\usepackage[locale=DE]{siunitx}
\usepackage{siunitx}
\usepackage[portuguese]{babel}
\usepackage{graphicx}
\usepackage{float}
\usepackage{caption}
\usepackage{subcaption}
\usepackage{amsmath}
\usepackage{url}
\usepackage{xcolor}
\usepackage{float}
\usepackage{pdfpages}
\usepackage{wrapfig}
\usepackage{diagbox}

\usepackage[colorlinks = true,
linkcolor = blue,
urlcolor  = blue,
citecolor = blue,
anchorcolor = blue]{hyperref}
\usepackage[margin=1in]{geometry}


\title{PL8 - Caracterização de materiais através de Ultra-sons}
\author{Miguel Rangel Dias, turma PL3, Grupo 1}

\begin{document} 
	
	\maketitle
	
	
	
	\begin{abstract}
		No contexto da unidade curricular de Laboratórios de Física III, trabalho PL7, utilizou-se um controlador \textit{Ultrasonics echoscope PHYWE} conjuntamente com o \textit{software} de visualização \textit{MeasureUltraEcho}. Nesse âmbito, estudou-se o funcionamento de transdutores piezo-elétricos na geração e captação de ondas sonoras. Determinou-se, através da análise de intervalos entre reflexões, a velocidade de ultra-sons no acrílico e avaliou-se a sua atenuação, determinando experimentalmente estes valores. Ainda com esta tecnologia, estudou-se em detalhe a estrutura interna de um bloco de acrílico com imperfeições, a profundidade e diâmetro das mesmas. Comparou-se a informação fornecida pelos varrimentos \textit{A-scan} e \textit{B-scan} e a resolução dada por diferentes frequências.
	\end{abstract}
	
	\section{Objetivos}
	O objetivo da experiência desenvolvida foi:
	\begin{itemize}
		\item Compreender o funcionamento de transdutores piezo-elétricos na geração e captação de ondas sonoras.
		\item Estudar a atenuação de ultra-sons no acrílico. 
		\item Determinar experimentalmente a velocidade do som no acrílico.
	\end{itemize}
	
	\section{Introdução teórica}
	\subsection{Refletância e transmitância}
	A compreensão de transdutores piezo-elétricos requer, primeiramente, um breve estudo de ondas acústicas.\\
	Relembram-se as expressões para a refletância $R$ na fronteira entre dois materiais 1 e 2, em função das impedâncias $Z_1, Z_2$:

	\begin{equation}
		R = \left(\frac{Z_1 - Z_2}{Z_1 + Z_2}\right)^2
		\label{refletância}
	\end{equation}
	A transmitância, por sua vez, é dada simplesmente por $T = 1-R$
	\begin{equation}
		T = \frac{4 Z_1 Z_2}{(Z_1 + Z_2)^2}
		\label{trasmitância}
	\end{equation}
	
	Sendo $Z \equiv \rho v$, usam-se valores de referência para o acrílico, ar e água \footnote{Os valores foram retirados de \url{https://signal-processing.com/table.php}.}.
	
	\begin{table}[H]
		\centering
		\begin{tabular}{|c|c|c|c|}
			\toprule
			\textbf{Parâmetros} & \textbf{Acrílico} & \textbf{Água} & \textbf{Ar} \\
			\midrule
			$Z (\unit{kg.m^{-2}.s^{-1}})$ & \num{3.26e6} & \num{1.476e6} & 385 \\
			\midrule
			Atenuação (\unit{dB .cm^{-1}}) & 6.4 & -- &  --\\
			\midrule
			Velocidade (\unit{m.s^{-1}}) & 2750 & -- & --\\
			\hline
		\end{tabular}
		\caption{Valores de referência para estudo prévio.}
	\end{table}
	Com base nestes valores\footnote{\textbf{Estes valores são meramente indicativos}, dependendo da frequência da onda e produção do material podem variar (estando, de qualquer modo, nesta ordem de grandeza).}, obtém-se:
	\begin{align*}
		R_{\text{ar - acrílico}} &\approx 0.9995 \\
		R_{\text{ar - água}} &\approx 0.9990 \\
		R_{\text{água - acrílico}} &\approx 0.142
	\end{align*}
	Daqui conclui-se que reflexões entre \textit{ar - água} ou \textit{ar - acrílico} são, geralmente, desprezáveis.
	
	\subsection{Atenuação}
	O modelo teórico da atenuação de uma onda acústica num meio é dado por um coeficiente de extinção característico $\mu$. Em particular, sendo a amplitude da onda $A_0$ em $s=0$, ao longo do seu percurso $A_s$ decrescerá com 
	\begin{equation*}
		A_s = A_0 \exp(-\mu s)
		\label{eq:Atenu}
	\end{equation*}
	Se nos referirmos à energia da onda (intensidade, $\langle A^2 \rangle_t$):
	\begin{equation}
		I_s = I_0 \exp(-2\mu s)
	\end{equation}
	
	\subsection{Transdutor piezo-elétrico}
	Através de um impulso num material cerâmico piezo-elétrico, geram-se ondas sonoras. O raciocínio inversa explica a sua deteção pelo mesmo. Uma grandeza relevante é o \textit{time of flight}: o intervalo de tempo entre a emissão da onda até à sua deteção (assumindo que ela retorna por reflexão).
	Neste caso, para um meio de profundidade $s$ e um \textit{time of flight} $\Delta t$, denota-se que: 
	\begin{equation}
		s = \frac{v}{2 \Delta t}
		\label{eqVandS}
	\end{equation}
	Pelo mesmo princípio, sabendo $s$ e $\Delta t$ pode-se determinar facilmente a velocidade da onda no meio $v$, tomando em atenção que \textbf{existem reflexões iniciais na interface de acoplamento}.\\
	
	O objetivo deste trabalho experimental é, portanto, utilizar estes conceitos-chave agora introduzidos para determinar as grandezas físicas referidas.

	\section{Procedimento experimental}
	
	\subsection{Material usado}
	\begin{itemize}
		\item Controlador \textit{Ultrasonics echoscope PHYWE}
		\item \textit{Software MeasureUltraEcho} (no computador do laboratório)
		\item Quatro cilindros de acrílico, com alturas diferentes
		\item Bloco paralelipipédico com estrutura a ser estudada (buracos de ar diferentes no seu interior)
		\item Paquímetro
		\item Sondas ultra-sónicas (1MHz e 2MHz) da \textit{PHYWE}
		\item Conta-gotas com água 
	\end{itemize}
	
	
	\subsection{Método experimental}
	Registaram-se as alturas dos três cilindros de menor altura. Colocou-se cada um assente numa das suas faces circulares, ficando outra voltada pra cima - nessa, será feita o acoplamento com a sonda. Pondo uma gota de água nesta face (para melhor acoplamento), encostou-se o transdutor piezo-elétrico para se observar no \textit{MeasureUltraEcho}, modo \textit{A-scan} (no computador), um gráfico de sinais detetados em função do tempo. Guardando estes dados em ficheiros ``.csv'', determinaram-se as grandezas requeridas na sua análise. \\
	Para a segunda parte (bloco paralelipipédico, esboço na figura \ref{fig:modeloimperfeicoes}), usaram-se as sondas para estudar a sua estrutura interna. Em concreto, colocou-se a sonda diretamente na vertical de cada um dos pontos de interesse - A, B e C. Fez-se este processo para face superior e, seguidamente, a face inferior (com ambas as sondas). Como já visto, a reflância entre \textit{ar-acrílico} é praticamente total, pelo que é justificado considerar que a onda embatente nos defeitos do bloco paralelipipédico têm um reflexão bem definida, a uma profundidade demarcada pelo \textit{software MeasureUltraEcho}. Comparando as profundidas das imperfeições encontradas numa orientação e noutra, estimamos os diâmetros. \\
	Por fim, usando \textit{B-scan}, verificou-se a formação de uma imagem de profundidade (imagem \ref{fig:bscan}).
	
	\begin{figure}[H]
		\centering
		\includegraphics[width=1\linewidth]{../BscanPic/modeloImperfeicoes}
		\caption{Esboço do bloco de acrílico com imperfeições.}
		\label{fig:modeloimperfeicoes}
	\end{figure}
	
	\begin{figure}[H]
		\centering
		\includegraphics[width=1\linewidth]{../BscanPic/bscan}
		\caption{Imagem retirada com \textit{B-scan}.}
		\label{fig:bscan}
	\end{figure}
	
	
	\section{Análise de dados}
	\subsection{Velocidade do som no acrílico}
	
	Para esta secção utilizaram-se ambas as sondas ultra-sónicas (1\unit{\mega\hertz} e 2\unit{\mega\hertz}). Primeiramente, mediu-se com recurso ao paquímetro as alturas dos cilindros:
	$$
	\begin{cases}
		h_1 = \SI{18.90(5)}{\milli\metre}\\
		h_2 = \SI{40.00(5)}{\milli\metre}\\
		h_3 = \SI{81.45(5)}{\milli\metre}\\
		h_4 = \SI{120.55(5)}{\milli\metre}\\
	\end{cases}
	$$
	Para cada um deles, foi feita a montagem referida, tendo sido adicionado algum ganho para melhor visualização gráfica. Guardaram-se os dados referentes a cada cilindro e frequência de sonda (gráficos \ref{fig:cilf1} e \ref{fig:cilf2}).
	
	O primeiro pico de cada gráfico (quase não visível na escala usada) foi desprezado como sendo uma reflexão entre o transdutor e a sua camada protetora. Tomando os picos marcados, determinou-se um intervalo médio $\Delta t$ que os separa. Pelas alturas dos cilindros e expressão \ref{eqVandS}, obtêm-se diferentes velocidades do som. Estes resultados estão apresentados na tabela \ref{tab:velocidade} - sublinha-se que \textbf{há uma diferença notável entre a velocidade para cada frequência}.
	
	
	
	\begin{table}[H]
		\centering
		\begin{tabular}{|c|c|c|}
			\toprule
			\diagbox{\textbf{Cilindro}}{\textbf{Sonda}} & f = 1 \unit{\mega\hertz} & f = 2 \si{\mega\hertz} \\
			\midrule
			$v_1 / \unit{m.s^{-1}}$ & \num{2729(40)} & 2754(40) \\
			$v_2 / \unit{m.s^{-1}}$ & 2706(19) & 2762(19) \\
			$v_3 / \unit{m.s^{-1}}$ & 2757(10) & 2767(10) \\
			\midrule
			\textbf{$v_{\textit{médio}} / \unit{m.s^{-1}}$} & 2731(15) & 2761(15) \\
			\bottomrule
		\end{tabular}
		\caption{Velocidade do som no acrílico para diferentes cilindros e frequências e respetivas incertezas (cobertura 100\%).}
		\label{tab:velocidade}
	\end{table}
	
	A incerteza de cada velocidade é dada tomando os termos da eq. \ref{eqVandS} ($\Delta t$ como intervalo entre duas reflexões $s$ as alturas dos cilindros) por (detalhes na secção \ref{apendice_u(v)}):
	 \begin{equation}
	 	u(v) = v\sqrt{\left(\frac{u(h)}{h}\right)^2 + \left(\frac{u(\Delta t)}{\Delta t}\right)^2}
	 	\label{eq:u(v)}
	 \end{equation}
	 em que se usa a seguinte notação: $u(x)$ é uma incerteza da grandeza $x$. Por conveniência, trabalha-se geralmente com cobertura $\approx 100\%$ ($3 \times \sigma_x$).
	
	
	
	
	\subsection{Atenuação de onda no acrílico}
	Tomando os dados (imagens \ref{fig:cilf1} e \ref{fig:cilf2}), fez-se o inverso do ganho para obter dados com a amplitude real da onda. Pelo facto de se ter atenuação demasiado elevada (por se ter um cilindro longo, por exemplo), para a maioria dos ensaios não se tinha mais que 2 pontos\footnote{Excluindo o primeiro pico, que não foi usado pela ambiguidade em torno da energia associada à sua amplitude; i.e., em torno da sua região há também reflexão na camada protetora.} notáveis o suficiente para efetuar o ajuste exponencial. Assim, a atenuação pode ser calculada apenas para um dos ensaios, em particular o cilindro 1, com frequência $f = \SI{2}{\hertz}$. Este pode ser visto no gráfico \ref{fig:atenu}.\\
	Pela eq. \ref{eq:Atenu}, vemos que o parâmetro ``k'' do ajuste no gráfico \ref{fig:atenu} corresponde a $\mu$, o coeficiente de extinção\footnote{$V$ é um parâmetro diretamente proporcional à amplitude de onda, portanto apesar de não se ter exatamente $A(s)$, este passo é válido.}. Temos portanto que $\mu = 33.77 \unit{m^{-1}}$. 
	
	É relevante passar para unidades de \unit{\decibel \metre^{-1}} (na secção \ref{A:units}):
	\begin{figure}[H]
		\centering
		\vspace{-5pt}
		$\mu = \SI{2.93(13)}{\decibel \cm^{-1}}$
		\caption{Valor empírico de $\mu$, com incerteza de cobertura $\approx 100\%$}
		\vspace{-5pt}
	\end{figure}
	
	\begin{figure}[H]
		\centering
		\vspace{-5pt}
		\includegraphics[width=1\linewidth]{../grafs/atenuation}
		\caption[]{Ajuste exponencial ao envelope de onda de um sinal proveniente de diversas reflexões, para o cilindro 1, $f = \SI{2}{\hertz}$; com incerteza de parâmetro $\sigma_{\mu} =1.50959$.}.
		\label{fig:atenu}
		\vspace{-5pt}
	\end{figure}
	
	
	\subsection{Imperfeições do bloco paralelipipédico}
	Para esta parte da atividade, utilizaram-se ambas as sondas.\\
	Percorreu-se

	Percorreu-se a face superior e inferior do bloco com ambas as sondas, obtendo diferentes gráficos de amplitude em função de profundidade (na secção \ref{A:grafsBloco}).  e os diâmetros destas imperfeições, comparando picos entre orientações diferentes conseguimos determinar os diâmetros dos defeitos representados por A,B e C na figura \ref{fig:modeloimperfeicoes}.
	
	
	Assumindo incerteza da posição dos picos $\SI{0.1}{\mm}$, obtém-se:
	\begin{table}
		\centering
		\begin{tabular}{|c|c|c|}
			\toprule
			\diagbox{\textbf{Sonda}}{\textbf{Defeito}} & f = 1 \unit{\mega\hertz} & f = 2 \si{\mega\hertz} \\
			\midrule
			$A / \unit{\mm}$ & \num{1.27(20)} & \num{1.57(20)} \\
			$B / \unit{\mm}$ & \num{2.07(20)} & \num{2.87(20)} \\
			\bottomrule
		\end{tabular}
		\caption{Diâmetro da imperfeição com incerteza associada (cobertura 100\%).}
		\label{tab:diamsAB}
	\end{table}	 	
	Denota-se que, para esta ordem de grandeza, a sonda $f = 1\unit{\mega\hertz}$ (com comprimento de onda correspondente $\lambda \approx \SI{2.75}{\mm}$) não tem resolução suficiente. Uma vez que C representa defeitos ainda mais pequenos (ver fig. \ref{fig:modeloimperfeicoes}), utilizou-se apenas a sonda de $f = 2\unit{\mega\hertz}$.\\
	Ampliando os gráficos \ref{fig:cy12} e \ref{fig:cy22} (que resulta nas fig. \ref{fig:zoomC} e \ref{fig:zoomCC}), temos que para cada ponto de C:
	\begin{table}[H]
		\centering
		\begin{tabular}{|c|c|}
			\toprule
			\textbf{Defeito} & f = 2 \si{\mega\hertz} \\
			\midrule
			(i) & \num{1.40(20)}\unit{\mm} \\
			(ii) & \num{2.40(20)}\unit{\mm} \\
			\bottomrule
		\end{tabular}
		\caption{Diâmetro da imperfeição com incerteza associada (cobertura 100\%).}
		\label{tab:diamsC}
	\end{table}	 	
	
	
	\section{Discussão de resultados}
	\paragraph{Velocidade do som no acrílico}
	Reiteram-se os valores obtidos na tabela \ref{tab:velocidade}, para as velocidades médias de ultra-sons no acrílico:
	\begin{align*}
		v(f=\SI{1}{\mega\hertz}) &= \num{2731(15)}\unit{\metre\per\second}\\
		v(f=\SI{2}{\mega\hertz}) &= \num{2761(15)}\unit{\metre\per\second} 
	\end{align*}
	Com base nestes valores, uma primeira abordagem diria que, como frequentemente observado noutros materiais, a velocidade de uma onda acústica no acrílico tem dependência com a frequência. No entanto, não seria esperado que a velocidade tivesse uma variação desta magnitude para um acrílico \footnote{Carlson, J. (2003). Frequency and temperature dependence of acoustic properties of polymers used in pulse-echo systems (Master’s thesis, Uppsala University). Uppsala University Publications. \url{https://www.diva-portal.org/smash/get/diva2:1007262/FULLTEXT01.pdf}}. Uma origem de erro possível poderá ser o cálculo de uma incerteza da média, assumindo cada ensaio como independente. Na realidade, como o mesmo erro sistemático e incerteza poderá estar presente em cada ensaio semelhante, este passo não poderá estar correto. Assumindo, portanto, uma incerteza da velocidade média majorada pela maior incerteza dos vários ensaios, teríamos, por excesso:
	\begin{align*}
		v(f=\SI{1}{\mega\hertz}) &= \num{2731(40)}\unit{\metre\per\second}\\
		v(f=\SI{2}{\mega\hertz}) &= \num{2761(40)}\unit{\metre\per\second} 
	\end{align*}
	Desde modo, cobriríamos facilmente uma gama de valores comuns expectáveis (em torno de $\approx 2750 \unit{\metre\per\second}$) sem chegar a contradições. \\
	Outra opção seria simplesmente considerar o cilindro mais longo como sendo aquele que fornece um valor mais provável de velocidade. Assim, ter-se-ia:
	\begin{align*}
		v(f=\SI{1}{\mega\hertz}) &= \num{2757(10)}\unit{\metre\per\second}\\
		v(f=\SI{2}{\mega\hertz}) &= \num{2767(10)}\unit{\metre\per\second} 
	\end{align*}
	que são valores perfeitamente coerentes e coincidentes com o valor expectável encontrado nas referências bibliográficas. 
	
	\paragraph{Estudo do bloco paralelipipédico}
	Para esta parte da atividade experimental, os resultados esperados pelo protocolo experimental eram:
	
	\begin{table}[H]
		\centering
		\begin{tabular}{|c|c|}
			\toprule
			Defeito & diâmetro / \unit{\mm}\\
			\midrule
			$A$ & \num{3}\\
			$B$ & \num{4}\\
			$C$ & \num{1.5} \\
			\bottomrule
		\end{tabular}
		\caption{Diâmetro da imperfeição pelo protocolo laboratorial.}
		\label{tab:diamsExp}
	\end{table}	 
	
	\section{Conclusão}
	
	Para a primeira parte da experiência obteve-se uma verificação nítida da lei do inverso dos quadrados para a propagação de luz de uma fonte pontual. A não linearidade observada nos resíduos foi explicada com a dissipação da luz na sua propagação no ar (à medida que a potência reduz, efeitos de dissipação tornam-se mais relevantes por comparação).\\
	
	Para a determinação dos coeficientes de absorção, obteve-se:
	\begin{equation*}
		\begin{cases}
			\alpha_{(1)} = \SI{2.349 \pm 0.0084}{\cm^{-1}}\\
			\alpha_{(2)} = \SI{1.837 \pm 0.0086}{\cm^{-1}}
		\end{cases}
	\end{equation*}
	Que se corroborou serem valores prováveis através de discussão com outros grupos.
	 
	
	\newpage
	\appendix       % declare the appendix
	\section{Apêndice}
	\label{apendice}
	
	\subsection{Gráficos de sinal no cilíndro de acrílico}
	
	\begin{figure}[H]
		\centering
		\vspace{-5pt}
		\includegraphics[width=0.8\linewidth]{../grafs/CilF1}
		\caption{Envelope de onda para o sinal obtido para acoplamento entre os cilindros e sonda de 1Mhz e respetivo ganho, em função do tempo.}
		\label{fig:cilf1}
		\vspace{-5pt}
	\end{figure}
	
	
	\begin{figure}[H]
		\centering
		\vspace{-5pt}
		\includegraphics[width=0.8\linewidth]{../grafs/CilF2}
		\caption[]{Envelope de onda para o sinal obtido para acoplamento entre os cilindros e sonda de 2Mhz e respetivo ganho, em função do tempo.}
		\label{fig:cilf2}
		\vspace{-5pt}
	\end{figure}


	\subsection{Propagação de incerteza para $v$}
	Cálculo auxiliar para a incerteza de $v = \frac{2h}{\Delta t}$:
	\label{apendice_u(v)}
	\begin{align*}
		u^2(v) &= \left(\frac{\partial v}{\partial \Delta t} u(\Delta t)\right)^2 + \left(\frac{\partial v}{\partial h} u(h)\right)^2 \\
		u^2(v) &= \left(\frac{-2h}{\Delta t^2} u(\Delta t)\right)^2 + \left(\frac{2}{\Delta t} u(h)\right)^2 \\
		u^2(v) &= v^2\left[\left(\frac{u(\Delta t)}{\Delta t} \right)^2 + \left(\frac{u(h)}{h} \right)^2\right]\\
		\Rightarrow u(v) &= v\sqrt{\left(\frac{u(\Delta t)}{\Delta t} \right)^2 + \left(\frac{u(h)}{h} \right)^2}
	\end{align*}
	
	
	\subsection{Conversão de unidades}
	\label{A:units}
	Esta secção dedica-se à conversão $\unit{\m^{-1}} \to \unit{\decibel \m^{-1}}$.
	\begin{align*}
		10\log_{10}(\frac{P_0}{P}) &= \text{atenuação}(\unit{\decibel})\\
		20\log_{10}(\frac{A_0}{A}) &= \text{atenuação}(\unit{\decibel})\\
	\end{align*}
	Sabemos também que $\frac{A_0}{A} = \exp(\mu s)$, com $\mu$ em unidades de $L^{-1}$. Por comparação direta:
	\begin{align*}
		&20\log_{10}(e) \mu[\unit{\metre^{-1}}] \times s = \text{atenuação}(\unit{\decibel})\\
		&\Rightarrow \mu[\unit{\decibel.\m^{-1}}] = 20\log_{10}(e) \mu[\unit{\m^{-1}}] \approx 8.6859\, \mu[\unit{\m^{-1}}]
	\end{align*}
	
	
	\subsection{Gráficos para os dados do bloco paralelipipédico}
	\label{A:grafsBloco}	

		\begin{figure}[H]
		\centering
		\vspace{-5pt}
		\includegraphics[width=1\linewidth]{../grafs/Ay11}
		\caption{Gráfico da amplitude do sinal em função do tempo do ponto A, para $f = 1\unit{\mega\hertz}$ na primeira orientação.}
		\label{fig:ay11}
		\vspace{-5pt}
	\end{figure}
	
	\begin{figure}[H]
		\centering
		\vspace{-5pt}
		\includegraphics[width=1\linewidth]{../grafs/Ay21}
		\caption{Gráfico da amplitude do sinal em função do tempo do ponto A, para $f = 1\unit{\mega\hertz}$ na segunda orientação.}
		\label{fig:ay21}
		\vspace{-5pt}
	\end{figure}
	
	\begin{figure}[H]
		\centering
		\vspace{-5pt}
		\includegraphics[width=1\linewidth]{../grafs/By11}
		\caption{Gráfico da amplitude do sinal em função do tempo do ponto B, para $f = 1\unit{\mega\hertz}$ na primeira orientação.}
		\label{fig:by11}
		\vspace{-5pt}
	\end{figure}
	
	\begin{figure}[H]
		\centering
		\vspace{-5pt}
		\includegraphics[width=1\linewidth]{../grafs/By21}
		\caption{Gráfico da amplitude do sinal em função do tempo do ponto B, para $f = 1\unit{\mega\hertz}$ na segunda orientação.}
		\label{fig:by21}
		\vspace{-5pt}
	\end{figure}
	
	\begin{figure}[H]
		\centering
		\vspace{-5pt}
		\includegraphics[width=1\linewidth]{../grafs/Cy11}
		\caption{Gráfico da amplitude do sinal em função do tempo do ponto C, para $f = 1\unit{\mega\hertz}$ na primeira orientação.}
		\label{fig:cy11}
		\vspace{-5pt}
	\end{figure}
	
	\begin{figure}[H]
		\centering
		\vspace{-5pt}
		\includegraphics[width=1\linewidth]{../grafs/Cy21}
		\caption{Gráfico da amplitude do sinal em função do tempo do ponto C, para $f = 1\unit{\mega\hertz}$ na segunda orientação.}
		\label{fig:cy21}
		\vspace{-5pt}
	\end{figure}
	
	
	\begin{figure}[H]
		\centering
		\vspace{-5pt}
		\includegraphics[width=1\linewidth]{../grafs/Ay12}
		\caption{Gráfico da amplitude do sinal em função do tempo do ponto A, para $f = 2\unit{\mega\hertz}$ na primeira orientação.}
		\label{fig:ay12}
		\vspace{-5pt}
	\end{figure}
	
	\begin{figure}[H]
		\centering
		\vspace{-5pt}
		\includegraphics[width=1\linewidth]{../grafs/Ay22}
		\caption{Gráfico da amplitude do sinal em função do tempo do ponto A, para $f = 2\unit{\mega\hertz}$ na segunda orientação.}
		\label{fig:ay22}
		\vspace{-5pt}
	\end{figure}
	
	\begin{figure}[H]
		\centering
		\vspace{-5pt}
		\includegraphics[width=1\linewidth]{../grafs/By12}
		\caption{Gráfico da amplitude do sinal em função do tempo do ponto B, para $f = 2\unit{\mega\hertz}$ na primeira orientação.}
		\label{fig:by12}
		\vspace{-5pt}
	\end{figure}
	
	\begin{figure}[H]
		\centering
		\vspace{-5pt}
		\includegraphics[width=1\linewidth]{../grafs/By22}
		\caption{Gráfico da amplitude do sinal em função do tempo do ponto B, para $f = 2\unit{\mega\hertz}$ na segunda orientação.}
		\label{fig:by22}
		\vspace{-5pt}
	\end{figure}
	
	\begin{figure}[H]
		\centering
		\vspace{-5pt}
		\includegraphics[width=1\linewidth]{../grafs/Cy12}
		\caption{Gráfico da amplitude do sinal em função do tempo do ponto C, para $f = 2\unit{\mega\hertz}$ na primeira orientação.}
		\label{fig:cy12}
		\vspace{-5pt}
	\end{figure}
	
	\begin{figure}[H]
		\centering
		\vspace{-5pt}
		\includegraphics[width=1\linewidth]{../grafs/Cy22}
		\caption{Gráfico da amplitude do sinal em função do tempo do ponto C, para $f = 2\unit{\mega\hertz}$ na segunda orientação.}
		\label{fig:cy22}
		\vspace{-5pt}
	\end{figure}
	
	\begin{figure}[H]
		\centering
		\vspace{-5pt}
		\includegraphics[width=1\linewidth]{../grafs/zoomC}
		\caption{Gráfico da amplitude do sinal em função do tempo do ponto C, para $f = 2\unit{\mega\hertz}$ na primeira orientação, ampliado na zona de interesse.}
		\label{fig:zoomC}
		\vspace{-5pt}
	\end{figure}
	
	\begin{figure}[H]
		\centering
		\vspace{-5pt}
		\includegraphics[width=1\linewidth]{../grafs/zoomCC}
		\caption{Gráfico da amplitude do sinal em função do tempo do ponto C, para $f = 2\unit{\mega\hertz}$ na segunda orientação, ampliado na zona de interesse.}
		\label{fig:zoomCC}
		\vspace{-5pt}
	\end{figure}

	
\end{document}
