\documentclass[a4paper,10pt,twocolumn]{article}

\usepackage{booktabs}
\usepackage[locale=DE]{siunitx}
\usepackage{siunitx}
\usepackage[portuguese]{babel}
\usepackage{graphicx}
\usepackage{float}
\usepackage{caption}
\usepackage{subcaption}
\usepackage{amsmath}
\usepackage{url}
\usepackage{xcolor}
\usepackage{float}
\usepackage{pdfpages}
\usepackage{wrapfig}

\usepackage[colorlinks = true,
linkcolor = blue,
urlcolor  = blue,
citecolor = blue,
anchorcolor = blue]{hyperref}
\usepackage[margin=1in]{geometry}


\title{PL8 - Caracterização de materiais através de Ultra-sons}
\author{Miguel Rangel Dias, turma PL3, Grupo 1}

\begin{document} 
	
	\maketitle
	
	
	
	\begin{abstract}
		No contexto da unidade curricular de Laboratórios de Física III, trabalho PL7, utilizou-se um controlador \textit{Ultrasonics echoscope PHYWE} conjuntamente com o \textit{software} de visualização \textit{MeasureUltraEcho}. Nesse âmbito, estudou-se o funcionamento de transdutores piezo-elétricos na geração e captação de ondas sonoras. Determinou-se, através da análise de intervalos entre reflexões, a velocidade de ultra-sons no acrílico e avaliou-se a sua atenuação, determinando experimentalmente estes valores. Comparou-se a informação fornecida pelos varrimentos \textit{A-scan} e \textit{B-scan} e a resolução dada por diferentes frequências.
	\end{abstract}
	
	\section{Objetivos}
	O objetivo da experiência desenvolvida foi:
	\begin{itemize}
		\item Compreender o funcionamento de transdutores piezo-elétricos na geração e captação de ondas sonoras.
		\item Estudar a atenuação de ultra-sons no acrílico. 
		\item Determinar experimentalmente a velocidade do som no acrílico.
	\end{itemize}
	
	\section{Introdução teórica}
	\subsection{Refletância e transmitância}
	A compreensão de transdutores piezo-elétricos requer, primeiramente, um breve estudo de ondas acústicas.\\
	Relembram-se as expressões para a refletância $R$ na fronteira entre dois materiais 1 e 2, em função das impedâncias $Z_1, Z_2$:

	\begin{equation}
		R = \left(\frac{Z_1 - Z_2}{Z_1 + Z_2}\right)^2
		\label{refletância}
	\end{equation}
	A transmitância, por sua vez, é dada simplesmente por $T = 1-R$
	\begin{equation}
		T = \frac{4 Z_1 Z_2}{(Z_1 + Z_2)^2}
		\label{trasmitância}
	\end{equation}
	
	Sendo $Z \equiv \rho v$, usam-se valores de referência para o acrílico, ar e água \footnote{Os valores foram retirados de \url{https://signal-processing.com/table.php}.}.
	
	\begin{table}[H]
		\centering
		\begin{tabular}{|c|c|c|c|}
			\toprule
			\textbf{Parâmetros} & \textbf{Acrílico} & \textbf{Água} & \textbf{Ar} \\
			\midrule
			$Z (\unit{kg.m^{-2}.s^{-1}})$ & \num{3.26e6} & \num{1.476e6} & 385 \\
			\midrule
			Atenuação (\unit{dB .cm^{-1}}) & 6.4 & -- &  --\\
			\midrule
			Velocidade (\unit{m.s^{-1}}) & 2750 & -- & --\\
			\hline
		\end{tabular}
		\caption{Valores de referência para estudo prévio.}
	\end{table}
	Com base nestes valores\footnote{\textbf{Estes valores são meramente indicativos}, dependendo da frequência da onda e produção do material podem variar (estando, de qualquer modo, nesta ordem de grandeza).}, obtém-se:
	\begin{align*}
		R_{\text{ar - acrílico}} &\approx 0.9995 \\
		R_{\text{ar - água}} &\approx 0.9990 \\
		R_{\text{água - acrílico}} &\approx 0.142
	\end{align*}
	Daqui conclui-se que reflexões entre \textit{ar - água} ou \textit{ar - acrílico} são, geralmente, desprezáveis.
	
	\subsection{Atenuação}
	O modelo teórico da atenuação de uma onda acústica num meio é dado por um coeficiente de extinção característico $\mu$. Em particular, sendo amplitude $A_0$ em $s=0$, a amplitude deste onda ao longo do seu percurso decrescerá com 
	\begin{equation*}
		A = A_0 \exp(-\mu s)
	\end{equation*}
	Se nos referirmos à energia da onda (intensidade, $\langle A^2 \rangle_t$):
	\begin{equation}
		I = I_0 \exp(-2\mu s)
	\end{equation}
	
	\subsection{Transdutor piezo-elétrico}
	Através de um impulso num material cerâmico piezo-elétrico, geram-se ondas sonoras. O raciocínio inversa explica a sua deteção pelo mesmo. Uma grandeza relevante é o \textit{time of flight}: o intervalo de tempo entre a emissão da onda até à sua deteção (assumindo que ela retorna por reflexão).
	No caso um meio de profundidade $s$, um \textit{time of flight} $\Delta t$, denota-se que: 
	\begin{equation}
		s = \frac{v}{2 \Delta t}
		\label{eqVandS}
	\end{equation}
	Pelo mesmo princípio, sabendo $s$ pode-se determinar facilmente a velocidade da onda no meio $v$, tomando em atenção que \textbf{existem reflexões iniciais na interface de acoplamento}.
	
	
	\section{Procedimento experimental}
	
	\subsection{Material usado}
	\begin{itemize}
		\item Controlador \textit{Ultrasonics echoscope PHYWE}
		\item \textit{Software MeasureUltraEcho} (no computador do laboratório)
		\item Quatro cilindros de acrílico, com alturas diferentes
		\item Bloco paralelipipédico com estrutura a ser estudada (buracos de ar diferentes no seu interior)
		\item Paquímetro
		\item Sondas ultra-sónicas (1MHz e 2MHz) da \textit{PHYWE}
		\item Conta-gotas com água 
	\end{itemize}
	
	
	\subsection{Metodologia}
	Registaram-se as alturas dos três cilindros de menor altura. Colocou-se cada um assente numa das suas faces circulares, ficando outra voltada pra cima - nessa, será feita o acoplamento com a sonda. Pondo uma gota de água nesta fase (para melhor acoplamento), encostou-se o transdutor piezo-elétrico para se observar no \textit{MeasureUltraEcho}, no computador, um gráfico de sinais detetados em função do tempo. Guardando estes dados em ficheiros ``.csv'', determinaram-se as grandezas requeridas na sua análise. Para a segunda parte (bloco paralelipipédico), usaram-se as sondas para estudar a sua estrutura interna. 
	
	\section{Análise de dados}
	\subsection{Velocidade do som no acrílico}
	Para esta secção utilizaram-se ambas as sondas ultra-sónicas (1MHz e 2MHz). Primeiramente, mediu-se com recurso ao paquímetro as alturas dos cilindros:
	$$
	\begin{cases}
		h_1 = \SI{18.90(5)}{\milli\metre}\\
		h_2 = \SI{40.00(5)}{\milli\metre}\\
		h_3 = \SI{81.45(5)}{\milli\metre}\\
		h_4 = \SI{120.55(5)}{\milli\metre}\\
	\end{cases}
	$$
	Para cada um deles, foi feita a montagem referida, tendo sido adicionado algum ganho para melhor visualização gráfica. Guardaram-se os dados referentes a cada cilindro e frequência de sonda (gráficos \ref{fig:cilf1} e \ref{fig:cilf2}).
	\begin{figure}[H]
		\centering
		\vspace{-5pt}
		\includegraphics[width=0.8\linewidth]{../grafs/CilF1}
		\caption{Envelope de onda para o sinal obtido para acoplamento entre os cilindros e sonda de 1Mhz e respetivo ganho, em função do tempo.}
		\label{fig:cilf1}
		\vspace{-5pt}
	\end{figure}
	
	O primeiro pico de cada gráfico foi desprezado como sendo uma reflexão entre o transdutor e a sua camada protetora. Tomando os picos marcados, determinou-se um intervalo médio $\Delta t$ que os separa. Pelas alturas dos cilindros e expressão \ref{eqVandS}, obtêm-se diferentes velocidades do som (tabela \ref{tab:velocidade} - sublinha-se que \textbf{há uma diferença notável entre a velocidade para cada frequência}).
	 
	\begin{figure}[H]
		\centering
		\vspace{-5pt}
		\includegraphics[width=0.8\linewidth]{../grafs/CilF2}
		\caption[]{Envelope de onda para o sinal obtido para acoplamento entre os cilindros e sonda de 2Mhz e respetivo ganho, em função do tempo.}
		\label{fig:cilf2}
		\vspace{-5pt}
	\end{figure}
	
	Tomando os termos da eq. \ref{eqVandS} e usando $\Delta t$ como intervalo entre duas reflexões, a incerteza de $v$ é dada por (detalhes na secção \ref{apendice_u(v)}):
	\begin{equation}
		u(v) = v\sqrt{\left(\frac{u(h)}{h}\right)^2 + \left(\frac{u(\Delta t)}{\Delta t}\right)^2}
		\label{eq:u(v)}
	\end{equation}
	em que se usa a notação: $u(x)$ é uma incerteza da grandeza $x$. Por conveniência, trabalha-se sempre com cobertura $\approx 100\%$ ($3 \times \sigma_x$).
	
	\begin{table}[H]
		\centering
		\begin{tabular}{|c|c|c|}
			\toprule
			\textbf{Cilindro} & f = 1 \unit{\mega\hertz} & f = 2 \si{\mega\hertz} \\
			\midrule
			$v_1$ & 2729.24 & 2754.10 \\
			$v_2$ & 2705.75 & 2761.80 \\
			$v_3$ & 2757.45 & 2766.87 \\
			\midrule
			\textbf{$v_{\textit{médio}}$} & 2730.81 & 2760.92 \\
			\bottomrule
		\end{tabular}
		\caption{Velocidade do som no acrílico para diferentes cilindros e frequências.}
		\label{tab:velocidade}
	\end{table}
	
	
	\begin{figure}[H]
		\centering
		\vspace{-5pt}
		\includegraphics[width=1\linewidth]{../grafs/atenuation}
		\caption[]{Ajuste exponencial ao envelope de onda de um sinal proveniente de diversas reflexões, para o cilindro 1, $f = \SI{2}{\hertz}$}.
		\label{fig:atenu}
		\vspace{-5pt}
	\end{figure}
	
	Tomando os dados (imagens \ref{fig:cilf1} e \ref{fig:cilf2}), fez-se o inverso do ganho para obter dados com a amplitude real da onda. Pelo facto de se ter atenuação demasiado elevada (pelo facto de se ter um cilindro longo, por exemplo), para a maioria dos ensaios não se tinha mais que 2 pontos\footnote{Excluindo o primeiro pico, que não foi usado pela ambiguidade em torno da energia associada à sua amplitude; i.e., em torno da sua região há também reflexão na camada protetora.} notáveis o suficiente para efetuar o ajuste exponencial. Assim, a atenuação pode ser calculada apenas para um dos ensaios (gráf. \ref{fig:atenu}), em particular o cilindro 1, com frequência $f = \SI{2}{\hertz}$.
	
	\section{Discussão de resultados}
	
	\subsection{Lei do inverso dos quadrados}
	
	
	\section{Conclusão}
	
	Para a primeira parte da experiência obteve-se uma verificação nítida da lei do inverso dos quadrados para a propagação de luz de uma fonte pontual. A não linearidade observada nos resíduos foi explicada com a dissipação da luz na sua propagação no ar (à medida que a potência reduz, efeitos de dissipação tornam-se mais relevantes por comparação).\\
	
	Para a determinação dos coeficientes de absorção, obteve-se:
	\begin{equation*}
		\begin{cases}
			\alpha_{(1)} = \SI{2.349 \pm 0.0084}{\cm^{-1}}\\
			\alpha_{(2)} = \SI{1.837 \pm 0.0086}{\cm^{-1}}
		\end{cases}
	\end{equation*}
	Que se corroborou serem valores prováveis através de discussão com outros grupos.
	\onecolumn 
	Denota-se, por fim, a relevância da experiência efetuada para melhor compreensão de um dos fenómenos óticos essencial - a lei do inverso dos quadrados. 
	
	\appendix       % declare the appendix
	\section{Apêndice}
	\label{apendice}
	\subsection{Propagação de incerteza para $v$}
	Cálculo auxiliar para a incerteza de $v = \frac{2h}{\Delta t}$:
	\label{apendice_u(v)}
	\begin{align*}
		u^2(v) &= \left(\frac{\partial v}{\partial \Delta t} u(\Delta t)\right)^2 + \left(\frac{\partial v}{\partial h} u(h)\right)^2 \\
		u^2(v) &= \left(\frac{-2h}{\Delta t^2} u(\Delta t)\right)^2 + \left(\frac{2}{\Delta t} u(h)\right)^2 \\
		u^2(v) &= v^2\left[\left(\frac{u(\Delta t)}{\Delta t} \right)^2 + \left(\frac{u(h)}{h} \right)^2\right]\\
		\Rightarrow u(v) &= v\sqrt{\left(\frac{u(\Delta t)}{\Delta t} \right)^2 + \left(\frac{u(h)}{h} \right)^2}
	\end{align*}
	
	
\end{document}